\documentclass{article}
\usepackage{graphicx} % Required for inserting images
\usepackage{geometry}
\graphicspath{./images/}

\title{{\textbf{QBS 103 Final Project in \LaTeX}}}
\author{{\textbf{{\textit{Ray Yu}}}}}
\date{August 2023}

\begin{document}

\maketitle

\tableofcontents

\newpage

\section{Introduction of {\textit{BPI}}}
This gene encodes a lipopolysaccharide binding protein. It is associated with {\textbf{human neutrophil granules}} and has antimicrobial activity against gram-negative organisms.\\\\
{\textbf{$BPI$}} was initially identified in neutrophils, but is found in other tissues including the epithelial lining of mucous membranes. It is an endogenous antibiotic protein with potent killing activity against Gram-negative bacteria. It binds to compounds called lipopolysaccharides produced by Gram-negative bacteria. Lipolysaccharides are potent activators of the immune system; however, {\textbf{$BPI$}} at certain concentrations can prevent this activation.\cite{bpigene2}\\\\
This gene is implicated in {\textbf{Crohn's disease}}, which a chronic disease that causes inflammation in a person's digestive tract. It's also implicated in severe acute respiratory syndrome, which could result in severe breathing problems if infected with COVID-19.\cite{bpigene}

\section{Table for Covariates}
\begin{table}[h]
This table {\textbf{\cite{tablecontent}}} shows the covariates summary of the $BPI$ gene expression. There are two continuous covariates: age and follow up days which the mean and standard deviation is calculated. There are four categorical covariates: sex, mechanical ventilation, ICU status, and COVID-19 status which the percentage of the categories are calculated.
\caption{\label{tab:unnamed-chunk-2}Summary Table for Covariates \cite{tablecontent}}
\centering
\begin{tabular}[t]{l|l|l|l}
\hline
{\textbf{Male}} & {\textbf{72}} & {\textbf{Female}} & {\textbf{50}}\\
\hline
\hline
{\textbf{Age: mean (sd)}} & {\textbf{\textbf{62.28 (14.41)}}} & {\textbf{Age: mean (sd)}} & {\textbf{\textbf{59.30 (17.92)}}}\\
\hline
\hline
{\textbf{Mechanical Ventilation: n(\%)}} & {\textbf{\textbf{}}} & {\textbf{Mechanical Ventilation: n(\%)}} & {\textbf{\textbf{}}}\\
\hline
\hline
{\hspace{1em}Yes} & {\textbf{49}} & {\hspace{1em}Yes} & {\textbf{32}}\\
\hline
{\hspace{1em}No} & {\textbf{51}} & {\hspace{1em}No} & {\textbf{68}}\\
\hline
\hline
{\textbf{Follow Up Days: mean (sd)}} & {\textbf{\textbf{22.17 (17.04)}}} & {\textbf{Follow Up Days: mean (sd)}} & {\textbf{\textbf{26.24 (16.48)}}}\\
\hline
\hline
{\textbf{ICU Status: n(\%)}} & {\textbf{\textbf{}}} & {\textbf{ICU Status: n(\%)}} & {\textbf{\textbf{}}}\\
\hline
\hline
{\hspace{1em}Yes} & {\textbf{57}} & {\hspace{1em}Yes} & {\textbf{48}}\\
\hline
{\hspace{1em}No} & {\textbf{43}} & {\hspace{1em}No} & {\textbf{52}}\\
\hline
{\textbf{COVID-19 Status: n(\%)}} & {\textbf{\textbf{}}} & {\textbf{COVID-19 Status: n(\%)}} & {\textbf{\textbf{}}}\\
\hline
\hline
{\hspace{1em}COVID-19} & {\textbf{85}} & {\hspace{1em}COVID-19} & {\textbf{74}}\\
\hline
{\hspace{1em}NON COVID-19} & {\textbf{15}} & {\hspace{1em}NON COVID-19} & {\textbf{26}}\\
\hline
\hline
{\textbf{Charlson Score: mean (sd)}} & {\textbf{\textbf{3.39 (2.49)}}} & {\textbf{Charlson Score: mean (sd)}} & {\textbf{\textbf{3.54 (2.47)}}}\\
\end{tabular}
\cite{bpigene}
\end{table}

\begin{figure}[h]
    \centering
    \includegraphics[width = \textwidth]{SummaryTable.jpg}
    \caption{Summary Table}
    \label{fig:Summary}
\end{figure}


\newpage
Summary Table {\textbf{[Figure \ref{fig:Summary}]}} as appeared in R studio.

\section{References}

\bibliographystyle{plain}
\bibliography{references}

\newpage
\section{Figures}
The Histogram {\textbf{[Figure \ref{fig:Hist}]}} plots the $BPI$ gene expression against the frequency that appears within participants. It shows a positive skewed graph, where most people have less than 100 unit of $BPI$ gene expressions with very few outliners that have more than 200 unit of $BPI$ gene expressions.\\
\begin{figure}[h]
    \centering
    \includegraphics[width = \textwidth]{Histogram.jpg}
    \caption{Histogram for $BPI$ Gene}
    \label{fig:Hist}
\end{figure}

\newpage
The Scatter plot {\textbf{[Figure \ref{fig:Scatter}]}} plots the $BPI$ gene expression against the age (in years). There is not many significant correlations between the age and the gene expressions, however, we can see that the average $BPI$ gene expression for participants is below 100 (approximately 50) as indicated by the yellow line.\\
\begin{figure}[h]
    \centering
    \includegraphics[width = \textwidth]{Scatter.jpg}
    \caption{Scatter plot for $BPI$ Gene Expression vs Age}
    \label{fig:Scatter}
\end{figure}

\newpage
The following two box plots {\textbf{[Figure \ref{fig:BoxMV}]}} {\textbf{[Figure \ref{fig:BoxS}]}} plots the mechanical ventilation or sex against the $BPI$ gene expression with the color separated by sex or mechanical ventilation. People with higher gene expressions tend to use mechanical ventilation since it causes severe problems when infected with COVID-19. There is no substantial correlation between sex and $BPI$ gene expression.\\
\begin{figure}[h]
    \centering
    \includegraphics[width = \textwidth]{Box_MV.jpg}
    \caption{Box plot of $BPI$ Gene Expression vs Mechanical Ventilation}
    \label{fig:BoxMV}
\end{figure}

\newpage
\begin{figure}[h]
    \centering
    \includegraphics[width = \textwidth]{Box_S.jpg}
    \caption{Box plot of $BPI$ Gene Expression vs Sex}
    \label{fig:BoxS}
\end{figure}

\newpage

The area plot {\textbf{[Figure \ref{fig:Area}]}} plots the age against $BPI$ gene expression with the color separated by mechanical ventilation. As seen in the plot, when participants are younger, more participants do not need to use the mechanical ventilation, but as the age gets older, less and less people can do without mechanical ventilation.
\begin{figure}[h]
    \centering
    \includegraphics[width = \textwidth]{Area.jpg}
    \caption{Area plot of $BPI$ Gene Expression vs Age}
    \label{fig:Area}
\end{figure}

\newpage
The two heat map {\textbf{[Figure \ref{fig:Heat_L}]}} plots the chosen 11 genes vs all the participants while the heat map {\textbf{[Figure \ref{fig:Heat_S}]}} plots the chosen 11 genes vs 20 participants. In {\textbf{[Figure \ref{fig:Heat_L}]}}, we can see that most people are between low to medium levels of genes with a few exceptions in {\textit{AADACL4}} and {\textit{AADACL3}} that are extremely high to a gene level of 45 unit. Both heat maps shows clustered columns and rows and the tracking bars with two categorical covariates: sex and mechanical ventilation.
\begin{figure}[h]
    \centering
    \includegraphics[width = \textwidth]{Heatmap_L.jpg}
    \caption{Heat map: Gene vs All Participants}
    \label{fig:Heat_L}
\end{figure}

\newpage
\begin{figure}[h]
    \centering
    \includegraphics[width = \textwidth]{Heatmap_S.jpg}
    \caption{Heat map: Gene vs 20 Participants}
    \label{fig:Heat_S}
\end{figure}


\end{document}
